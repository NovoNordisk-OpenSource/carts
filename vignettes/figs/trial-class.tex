\documentclass[tikz,margin=2mm]{standalone}
% \documentclass[border=0mm]{standalone}
\usepackage{tikz}
\usepackage{amsmath}
\usepackage[utf8]{inputenc}
\usepackage[T2A, T1]{fontenc}



\newcommand{\E}{\mathbb{E}}
\newcommand{\var}{\mathbb{V}\!\!\operatorname{ar}}
\newcommand{\pr}{\mathbb{P}}
\newcommand{\distnb}{\operatorname{NB}}
\newcommand{\distpois}{\operatorname{Pois}}
\newcommand{\distbern}{\operatorname{Bernoulli}}
\newcommand{\distgamma}{\Gamma}

\usetikzlibrary{arrows, calc, positioning, shapes.multipart}
\tikzstyle{arrow} = [thick,->,>=stealth]
\tikzstyle{line} = [draw, -latex']

\tikzstyle{proc} =%
rounded corners,
[rectangle split,
rectangle split parts=2,
rectangle split part fill={blue!20, none},
rectangle split part align={center, left},
text width=5cm, text centered,
draw=black]

\tikzstyle{method} =%
rounded corners,
[rectangle split,
rectangle split parts=2,
rectangle split part fill={red!20, none},
rectangle split part align={center, left},
text width=5cm, text centered,
draw=black]

\begin{document}

\begin{tikzpicture}[node distance = 1.5cm and 1.5cm]

\node (trial) [draw=black, rectangle, fill=orange!10, minimum width=5cm, minimum height=2cm] {\texttt{Trial} class};

\node (covar) [proc, below left=of trial]  {
    \textbf{\texttt{covariates}} \\
    \nodepart{two}
    \vspace*{-1.2em}
    \begin{flushleft}
      Function (arguments \texttt{n, ...}) that generates a list of \texttt{data.table}'s with covariates with each element representing a unique time-point.
    \end{flushleft}
  };

\node (outcome) [proc, above=of trial]  {
    \textbf{\texttt{outcome}} \\
    \nodepart{two}
    \vspace*{-1.2em}
    \begin{flushleft}
      Function (arguments \texttt{x}: covariate data.table, \texttt{...}: additional
    arguments defining parameters of the outcome model) that generates the outcome given covariates.
    \end{flushleft}
  };

\node (exclusion) [proc, above left=of trial]  {
    \textbf{\texttt{exclusion}} \\
    \nodepart{two}
    \vspace*{-1.2em}
    \begin{flushleft}
      Function (arguments \texttt{data}: data.table, \texttt{...}) that defines exclusion / inclusion criterions for the trial.
    \end{flushleft}
  };

\node (info) [proc, left=of trial]  {
    \textbf{\texttt{info}} \\
    \nodepart{two}
    \vspace*{-1.2em}
    \begin{flushleft}
      Optional string describing the trial simulation.
    \end{flushleft}
  };


\node (simulate) [method, above right=of trial]  {
    \textbf{\$\texttt{simulate}} \\
    \nodepart{two}
    \vspace*{-1.2em}
    \begin{flushleft}
    \textit{Simulate data from the specified trial design.}
      \textbf{Arguments} \texttt{n}: sample size, \text{...}: additional arguments controlling covariate and outcome parameters. \textbf{Returns} `data.table` as defined by the object initialization with default subject identifier `id` and observation period `num`.
    \end{flushleft}
  };

\node (run) [method, right=of trial]  {
    \textbf{\$\texttt{run}} \\
    \nodepart{two}
    \vspace*{-1.2em}
    \begin{flushleft}
    \textit{Simulate and estimate parameters several times.}
      \textbf{Arguments} \texttt{R}: replications, \texttt{estimators}: list of estimators of the trial target parameter, \texttt{...} additional arguments to the simulation routine. \textbf{Returns} an object with the Monte Carlo simulation results which can be analyzed with the \texttt{summary} method.
    \end{flushleft}
  };

\node (estimatepower) [method, below=of trial]  {
    \textbf{\$\texttt{estimate\_power}} \\
    \nodepart{two}
    \vspace*{-1.2em}
    \begin{flushleft}
    \textit{Estimate the statistical power of an estimator in a given scenario using Monte Carlo simulations.}
      \textbf{Arguments} \texttt{R}: number of replications, \texttt{estimators}: estimator to  consider,
     \texttt{summary.args}: arguments to the summary method that defines the statistical null hypothesis,
      \texttt{...} additional arguments to the simulation routine. \textbf{Returns} estimated power for the given trial scenario.
    \end{flushleft}
  };

\node (estimatesamplesize) [method, below right=of trial]  {
    \textbf{\$\texttt{estimate\_samplesize}} \\
    \nodepart{two}
    \vspace*{-1.2em}
    \begin{flushleft}
    \textit{Estimate the minimum sample size needed to achieve a desired statistical power of an estimator in a given scenario.}
      \textbf{Arguments} \texttt{estimator}, \texttt{power}, \texttt{...} \\
      \textbf{Returns} an integer with an attribute 'power' with the actual power estimate.
    \end{flushleft}
  };



\draw [arrow, dashed] (covar) -- (trial);
\draw [arrow, dashed] (info) -- (trial);
\draw [arrow, dashed] (exclusion) -- (trial);
\draw [arrow, dashed] (outcome) -- (trial);

\draw [arrow] (trial) -- (simulate);
\draw [arrow] (trial) -- (run);
\draw [arrow] (trial) -- (estimatepower);
\draw [arrow] (trial) -- (estimatesamplesize);

\end{tikzpicture}
\end{document}